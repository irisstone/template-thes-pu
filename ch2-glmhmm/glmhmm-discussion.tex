\section{Discussion}
\label{sec:glmhmm:discussion}

Our findings indicate that the opposing contributions of DMS pathways to movement are minimal in the absence of a decision (Fig. \ref{fig:glmhmm:1}), while the pathways provide large and opponent contributions to decision-making. Moreover, this contribution depends on the demands of a task (Fig. \ref{fig:glmhmm:2}), as the effect of inhibition is much larger during decisions that require evidence accumulation relative to control tasks with weaker cognitive requirements yet similar sensory features and motor requirements (Fig. \ref{fig:glmhmm:3}). The GLM–HMM further revealed that even within the evidence accumulation task, the contribution of DMS pathways to choice is not fixed. For example, DMS pathways have little contribution when mice pursue a strategy of repeating previous choices during the evidence accumulation task (Fig. \ref{fig:glmhmm:6}). Thus, together our findings imply that opposing contributions of DMS pathways to behavior are dependent on task demands and internal state.

\subsection{Cross-task differences in effects of dorsomedial striatum pathway inhibition}
\label{sec:glmhmm:discussion-1}

Our finding that DMS activity contributes to the evidence accumulation task, but not to task variants with weaker cognitive demands, is broadly consistent with previous work based on lesions, pharmacology, and recordings implicating DMS in short-term memory and the dynamic comparison of the value of competing options \cite{balleine_role_2007, yartsev_causal_2018, lau_value_2008, ding_separate_2012, akhlaghpour_dissociated_2016, donahue_distinct_2018, shin_differential_2018, delevich_choice_2020, frank_mechanisms_2012}. But then why have most previous optogenetic pathway-specific manipulations emphasized an opposing role for DMS pathways in the direct control of motor output \cite{kravitz_regulation_2010, roseberry_cell-type-specific_2016, bartholomew_striatonigral_2016, yttri_opponent_2016, lee_anatomically_2020, lee_activation_2016}, rather than on decision-making? Prior work has overwhelmingly relied on the synchronous activation of striatal pathways, as opposed to the inhibition used here. While DMS pathway activation is sufficient to bias movements such as spontaneous rotations, we observed relatively little impact of inhibition on behavior in the absence of a decision (Fig. \ref{fig:glmhmm:1}). Taken together with previous work, our findings may thus imply limits in the use of artificial activations in assessing striatal pathway function. Our results may also imply that DMS pathways would not necessarily display opposing correlates of movements \cite{parker_diametric_2018, london_coordinated_2018, cui_concurrent_2013, barbera_spatially_2016, sippy_cell-type-specific_2015, jin_basal_2014}, but rather opposing correlates of a decision process \cite{cui_asymmetrical_2021, tang_opposing_2021, yartsev_causal_2018, ding_separate_2012, donahue_distinct_2018, frank_mechanisms_2012}.

While our observations are consistent with the classic view of opposing contributions of striatal pathways to behavior \cite{alexander_functional_1990}, several prominent studies have instead challenged this view by reporting non-opposing behavioral effects of activating each pathway \cite{soares-cunha_activation_2016, cole_optogenetic_2018, vicente_direct_2016, tecuapetla_complementary_2016, geddes_optogenetic_2018, wang_activation_2018, peak_striatal_2020}. This may be because the pathways of a specific striatal subregion only exert opposing control on behavior in a specific context, for example, during cognitively demanding decision-making as shown here for the DMS, or during an interval timing task that requires the proactive suppression of actions, as shown for the dorsolateral striatum \cite{cruz_striatal_2020}. Along these lines, our comparison to NAc pathways, where inhibition produced weak effects on behavior in similar directions (Fig. \ref{fig:glmhmm:3}p), may imply that we have not discovered the context in which NAc pathways have opposing contributions to behavior.

We designed our T-maze tasks to have very similar sensory features and identical motor requirements, and yet very different cognitive demands, as assessed by task accuracy. That being said, the sensory features were not identical. Therefore, while unlikely, we cannot rule out that the subtle sensory differences contributed to the cross-task differences in the effect of pathway inhibition. A future direction would be to maintain an identical sensory environment across tasks and instead change the decision-making rule to be more cognitively demanding.


\subsection{Within-task changes in effects of dorsomedial striatum pathway inhibition}
\label{sec:glmhmm:discussion-2}

Complimenting our cross-task comparison, we reveal the new insight that mice occupy time-varying latent states within a single task and that the contribution of DMS pathways to choice depends on the internal state of mice. The application of a GLM–HMM was critical in uncovering this feature of behavior, allowing the unsupervised discovery of latent states that differ in how external covariates were weighted to influence a choice \cite{calhoun_unsupervised_2019, eldar_effects_2011, ahilan_forgetful_2018}. This provided two insights on the contributions of DMS pathways to behavior.

First, the impact of DMS inhibition was diminished when mice occupied a task-disengaged state in which choice history heavily predicted decisions, while conversely, the impact of DMS inhibition was accentuated when mice occupied a task-engaged state in which sensory evidence strongly influenced choice (Fig. \ref{fig:glmhmm:6}). This strengthens our conclusion from the cross-task comparison, which is that DMS pathways have a greater contribution to behavior when actively accumulating evidence toward a decision.

Second, mice occupied two similar task-engaged states that were modestly distinguished in overall accuracy (Fig. \ref{fig:glmhmm:6}g) and prominently distinguished by the influence of DMS inhibition on choice (Fig. \ref{fig:glmhmm:6}a–d). While transitions between these two states were relatively rare on the same day, there were days that included both states (Fig. \ref{fig:glmhmm:7}). The discovery of these two states leads to the intriguing suggestion that mice are capable of accumulating evidence toward a decision in at least two neurally distinct manners: one that depends on each DMS pathway (state 2), and another that does not (state 1). This may relate to demonstrations that neural circuits have substantial capacity for compensation to perturbations \cite{goshen_dynamics_2011, fetsch_focal_2018}, and our modeling approach may provide a new avenue for the identification of such compensatory mechanisms on relatively short timescales.

While our work focused on the three-state GLM–HMM, our conclusions do not depend on assuming exactly three states. In fact, the cross-validated log-likelihood of our data is higher for four states than three. Yet the conclusions from the four-state model were similar to those from the three-state model (compare Fig. \ref{fig:glmhmm:6}a,b to Extended Data Fig. \ref{fig:ap1:ext7}d,e), and additional gains in log-likelihood decrease for larger numbers of states. Nevertheless, it will be important for future work to compare the GLM–HMM framework used here, which assumes discrete states, to models that assume continuously varying states \cite{ashwood_mice_2022, roy_extracting_2021}.

Altogether, our studies provide new perspectives on the neural mechanisms by which DMS pathways exert opponent control over behavior, with particular emphasis on the importance of accounting for task demands, internal state and associated behavioral strategies when assessing neural mechanisms. To this end, we expect our behavioral and computational frameworks to be of broad utility in uncovering the neural substrates of decision-making in a wide range of settings.

