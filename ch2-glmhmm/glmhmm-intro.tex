\section{Introduction}
\label{sec:glmhmm:glmhmm-intro}

The striatum is composed of two principal outputs, the direct and indirect pathways, which are thought to exert opposing effects on behavior \cite{alexander_functional_1990}. In support of this view, many influential studies have shown that pathway-specific activation of the striatum produces opposing behavioral biases \cite{kravitz_regulation_2010, roseberry_cell-type-specific_2016, bartholomew_striatonigral_2016, bakhurin_opponent_2020, lobo_cell_2010, kravitz_distinct_2012,yttri_opponent_2016,tai_transient_2012,nonomura_monitoring_2018,lee_anatomically_2020,cui_asymmetrical_2021,tang_opposing_2021,parker_diametric_2018}. For example, direct or indirect pathway activation oppositely influences locomotion \cite{kravitz_regulation_2010,roseberry_cell-type-specific_2016,bartholomew_striatonigral_2016,parker_diametric_2018}, licking \cite{bakhurin_opponent_2020,lee_anatomically_2020,chen_direct_2021}, left/right rotations \cite{kravitz_regulation_2010,roseberry_cell-type-specific_2016,lee_anatomically_2020,lee_activation_2016}, repetition/cessation of activation-paired behaviors \cite{lobo_cell_2010,kravitz_distinct_2012,yttri_opponent_2016}, and left/right movements to report value-based decisions \cite{tai_transient_2012,tang_opposing_2021}.

Despite this pioneering work, it remains unresolved whether the endogenous activity of the two pathways provides opposing control over the generation of movements, or instead contributes to the cognitive process of deciding which movement to perform. This is in part because pathway-specific manipulations have disproportionately relied on artificial and synchronous activation, rather than inhibition of endogenous activity \cite{kravitz_regulation_2010, roseberry_cell-type-specific_2016, bartholomew_striatonigral_2016, bakhurin_opponent_2020, lobo_cell_2010, kravitz_distinct_2012,yttri_opponent_2016,tai_transient_2012,nonomura_monitoring_2018,lee_anatomically_2020,tang_opposing_2021}. The imbalance towards reports of activation suggests a wealth of negative results from inhibition, raising questions about the function of the endogenous activity, and whether it contributes to cognition. In fact, most previous pathway-specific activation studies have not used cognitively demanding tasks, making it difficult to dissociate a role in the decision towards a movement versus the generation of the movement itself \cite{kravitz_regulation_2010, roseberry_cell-type-specific_2016, bartholomew_striatonigral_2016, lobo_cell_2010, lee_anatomically_2020,parker_diametric_2018,lee_activation_2016}. In contrast, studies of the striatum that were not pathway-specific have instead focused on cognitively demanding behaviors \cite{london_coordinated_2018,balleine_role_2007,yartsev_causal_2018,lau_value_2008,ding_separate_2012,barnes_activity_2005,yin_dynamic_2009,akhlaghpour_dissociated_2016}. Taken together, this raises the possibility that striatal pathways exert opposing control of movement in the context of decision-making, rather than directly controlling motor output irrespective of cognition.

Thus, to determine if the contribution of endogenous activity in striatal pathways depends on cognition, we examined the effects of pathway-specific inhibition across a set of virtual reality tasks that had the same motor output and similar sensory features, but different cognitive requirements. This allowed us to ask if a task’s demands determined the effect of pathway-specific inhibition on behavior. Second, we used a latent state model to identify time-varying states within the same task. This allowed us to determine if the contribution of each pathway to behavior changed across time, even within the same task.

We found that inhibition of neither pathway produced a detectable influence on behavior as mice navigated a virtual corridor in the absence of a decision-making requirement. In contrast, pathway-specific inhibition produced strong and opposing biases on decisions based on the accumulation of evidence in a virtual T-maze \cite{pinto_accumulation--evidence_2018}, and had weaker effects on choice during less demanding task variants. Our latent state model further revealed that even within the evidence accumulation task, mice occupy different states across time that differ in the weighting of sensory evidence and trial history, as well as the extent that pathway-specific inhibition impacts choice. Thus, by comparing the effects of pathway-specific inhibition across behavioral tasks, and across time within a task, we conclude that both demands of the task and internal state of the mice determine whether striatal pathways exert strong and opposing control over behavior. 