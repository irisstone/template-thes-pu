\subsection{Cross-task differences in effects of dorsomedial striatum pathway inhibition}
\label{sec:glmhmm:discussion-1}

Our finding that DMS activity contributes to the evidence accumulation task, but not to task variants with weaker cognitive demands, is broadly consistent with previous work based on lesions, pharmacology, and recordings implicating DMS in short-term memory and the dynamic comparison of the value of competing options \cite{balleine_role_2007, yartsev_causal_2018, lau_value_2008, ding_separate_2012, akhlaghpour_dissociated_2016, donahue_distinct_2018, shin_differential_2018, delevich_choice_2020, frank_mechanisms_2012}. But then why have most previous optogenetic pathway-specific manipulations emphasized an opposing role for DMS pathways in the direct control of motor output \cite{kravitz_regulation_2010, roseberry_cell-type-specific_2016, bartholomew_striatonigral_2016, yttri_opponent_2016, lee_anatomically_2020, lee_activation_2016}, rather than on decision-making? Prior work has overwhelmingly relied on the synchronous activation of striatal pathways, as opposed to the inhibition used here. While DMS pathway activation is sufficient to bias movements such as spontaneous rotations, we observed relatively little impact of inhibition on behavior in the absence of a decision (Fig. \ref{fig:glmhmm:1}). Taken together with previous work, our findings may thus imply limits in the use of artificial activations in assessing striatal pathway function. Our results may also imply that DMS pathways would not necessarily display opposing correlates of movements \cite{parker_diametric_2018, london_coordinated_2018, cui_concurrent_2013, barbera_spatially_2016, sippy_cell-type-specific_2015, jin_basal_2014}, but rather opposing correlates of a decision process \cite{cui_asymmetrical_2021, tang_opposing_2021, yartsev_causal_2018, ding_separate_2012, donahue_distinct_2018, frank_mechanisms_2012}.

While our observations are consistent with the classic view of opposing contributions of striatal pathways to behavior \cite{alexander_functional_1990}, several prominent studies have instead challenged this view by reporting non-opposing behavioral effects of activating each pathway \cite{soares-cunha_activation_2016, cole_optogenetic_2018, vicente_direct_2016, tecuapetla_complementary_2016, geddes_optogenetic_2018, wang_activation_2018, peak_striatal_2020}. This may be because the pathways of a specific striatal subregion only exert opposing control on behavior in a specific context, for example, during cognitively demanding decision-making as shown here for the DMS, or during an interval timing task that requires the proactive suppression of actions, as shown for the dorsolateral striatum \cite{cruz_striatal_2020}. Along these lines, our comparison to NAc pathways, where inhibition produced weak effects on behavior in similar directions (Fig. \ref{fig:glmhmm:3}p), may imply that we have not discovered the context in which NAc pathways have opposing contributions to behavior.

We designed our T-maze tasks to have very similar sensory features and identical motor requirements, and yet very different cognitive demands, as assessed by task accuracy. That being said, the sensory features were not identical. Therefore, while unlikely, we cannot rule out that the subtle sensory differences contributed to the cross-task differences in the effect of pathway inhibition. A future direction would be to maintain an identical sensory environment across tasks and instead change the decision-making rule to be more cognitively demanding.

