\subsection{Within-task changes in effects of dorsomedial striatum pathway inhibition}
\label{sec:glmhmm:discussion-2}

Complimenting our cross-task comparison, we reveal the new insight that mice occupy time-varying latent states within a single task and that the contribution of DMS pathways to choice depends on the internal state of mice. The application of a GLM–HMM was critical in uncovering this feature of behavior, allowing the unsupervised discovery of latent states that differ in how external covariates were weighted to influence a choice \cite{calhoun_unsupervised_2019, eldar_effects_2011, ahilan_forgetful_2018}. This provided two insights on the contributions of DMS pathways to behavior.

First, the impact of DMS inhibition was diminished when mice occupied a task-disengaged state in which choice history heavily predicted decisions, while conversely, the impact of DMS inhibition was accentuated when mice occupied a task-engaged state in which sensory evidence strongly influenced choice (Fig. \ref{fig:glmhmm:6}). This strengthens our conclusion from the cross-task comparison, which is that DMS pathways have a greater contribution to behavior when actively accumulating evidence toward a decision.

Second, mice occupied two similar task-engaged states that were modestly distinguished in overall accuracy (Fig. \ref{fig:glmhmm:6}g) and prominently distinguished by the influence of DMS inhibition on choice (Fig. \ref{fig:glmhmm:6}a–d). While transitions between these two states were relatively rare on the same day, there were days that included both states (Fig. \ref{fig:glmhmm:7}). The discovery of these two states leads to the intriguing suggestion that mice are capable of accumulating evidence toward a decision in at least two neurally distinct manners: one that depends on each DMS pathway (state 2), and another that does not (state 1). This may relate to demonstrations that neural circuits have substantial capacity for compensation to perturbations \cite{goshen_dynamics_2011, fetsch_focal_2018}, and our modeling approach may provide a new avenue for the identification of such compensatory mechanisms on relatively short timescales.

While our work focused on the three-state GLM–HMM, our conclusions do not depend on assuming exactly three states. In fact, the cross-validated log-likelihood of our data is higher for four states than three. Yet the conclusions from the four-state model were similar to those from the three-state model (compare Fig. \ref{fig:glmhmm:6}a,b to Extended Data Fig. \ref{fig:ap1:ext7}d,e), and additional gains in log-likelihood decrease for larger numbers of states. Nevertheless, it will be important for future work to compare the GLM–HMM framework used here, which assumes discrete states, to models that assume continuously varying states \cite{ashwood_mice_2022, roy_extracting_2021}.

Altogether, our studies provide new perspectives on the neural mechanisms by which DMS pathways exert opponent control over behavior, with particular emphasis on the importance of accounting for task demands, internal state and associated behavioral strategies when assessing neural mechanisms. To this end, we expect our behavioral and computational frameworks to be of broad utility in uncovering the neural substrates of decision-making in a wide range of settings.

