\subsection{Inhibition of pathway-specific DMS activity is effective}
\label{sec:glmhmm:2.2.1}
We first sought to validate the effectiveness of halorhodopsin19 (NpHR)-mediated inhibition of indirect and direct striatal pathway activity in awake, head-fixed mice (Fig. ~\ref{fig:glmhmm:1}a and Extended Data Fig.~\ref{fig:ap1:ext1}a-b). Toward this end, we bilaterally delivered virus carrying Cre-dependent NpHR to the dorsomedial striatum (DMS) in transgenic mouse lines (A2a-Cre/D2R-Cre/D1R-Cre), which we verified to have high degrees of specificity and pentrance for each pathway (Supplementary Fig.~\ref{fig:ap1:supp1}). We confirmed that 532-nm (5mW) light delivery to the DMS through a tapered optical fiber produced rapid, sustained, and reversible inhibition of spiking in mice expressing NpHR in the indirect (Fig. ~\ref{fig:glmhmm:1}b and Extended Data Fig.~\ref{fig:ap1:ext1}c-e, n = 18/60, 30\% of neurons significantly inhibited) or direct pathway (Fig. ~\ref{fig:glmhmm:1}c and Extended Data Fig.~\ref{fig:ap1:ext1}f-h, n = 21/50, 42\% of neurons significantly inhibited). Moreover, we observed: (1) minimal excitation during illumination \cite{owen_thermal_2019,cruz_striatal_2020} (Extended Data Fig.~\ref{fig:ap1:ext1}d,g, left), (2) minimal effects on spiking upon laser offset (Extended Data Fig.~\ref{fig:ap1:ext1}d,g, right), indicating limited post-inhibitory rebound, and (3) stability in the efficacy of inhibition across time (Supplementary Fig. ~\ref{fig:ap1:supp2}). All together, our findings indicate that NpHR-mediated inhibition of DMS pathways is effective. 

