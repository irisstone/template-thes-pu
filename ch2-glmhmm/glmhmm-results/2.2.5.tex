\subsection{Little effect of nucleus accumbens pathway inhibition on choice}
\label{sec:glmhmm:2.2.5}

We next sought to determine whether opponent control of choice by striatal pathways during the evidence accumulation task was specific to the DMS, or if it extended to the ventral striatum. To this end, we delivered unilateral laser illumination to the nucleus accumbens (NAc) of mice expressing NpHR in the indirect or direct pathways (or non-opsin control mice), which was restricted to the cue region (0–200cm) of the evidence accumulation task (Fig. \ref{fig:glmhmm:3}l–p and Extended Data Fig. \ref{fig:ap1:ext4}j–l).

Providing a clear functional dissociation between DMS and NAc, effects of pathway-specific NAc inhibition on choice bias were significantly smaller than those observed with inhibition of DMS pathways (Extended Data Fig. \ref{fig:ap1:ext5}e,f; unpaired, two-tailed Wilcoxon rank-sum test of DMS versus NAc indirect pathway inhibition, $P=2.6 \times 10^{-4}$, $z=3.6$; of DMS versus NAc direct pathway inhibition, $P=1.8 \times 10^{-4}$, $z=-3.7$), and were also not significantly different from NAc control animals (Fig. \ref{fig:glmhmm:3}o,p). It is unlikely that this dissociation between DMS and NAc can be explained by greater coexpression of pathway-specific markers in the ventral versus dorsal striatum \cite{kupchik_coding_2015}, as both subregions exhibited equally low colocalization of D1R and D2R receptors (Supplementary Fig. \ref{fig:ap1:supp1}j–l).