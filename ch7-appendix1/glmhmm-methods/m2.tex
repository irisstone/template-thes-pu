\subsection{Surgical procedures}
\label{sec:ap1:m2}
All mice underwent sterile stereotaxic surgery to implant ferrule coupled optical fibers (Newport, 200 µM core, 0.37 NA) and a custom titanium headplate for head-fixation under isoflurane anesthesia (5\% induction, 1.5\% maintenance). Mice received a preoperative antibiotic injection of Baytril (5mg/kg, I.M.), as well as analgesia pre-operatively and 24-hours later in the form of meloxicam injections (2mg/kg, S.C.). A microsyringe pump controlling a 10µl glass syringe (Nanofill) was used to bilaterally deliver virus targeted to either the DMS (0.74 mm anterior, 1.5 mm lateral, -3.0 mm ventral) or the NAc (1.3 mm anterior, 1.2 mm lateral, -4.7 mm ventral). For optogenetic inhibition, the following viruses were used: AAV5-eF1a-DIO-eNpHR3.0-EYFP-WPRE-hGH (UPenn, 1.3 x 1013 parts/mL) or AAV5-eF1a-DIO-eNpHR3.0-EYFP-WPRE-hGH (PNI Viral Core, 2.2 x 1014 parts/mL, 1:5 dilution). For fluorescence in situ hybridization experiments, AAV5-eF1a-DIO-EYFP-hGHpA (PNI Viral Core, 6.0 x 1013 parts/mL) was used to label D1R+ and D2R+ neurons in D1R-Cre and A2A-Cre transgenic lines. In all experiments, virus was delivered at a rate of 0.2 µL/min for a total volume of 0.3-0.7 µL in the DMS, or 0.3-0.4 µL in the NAc. To accommodate patch fiber coupling, optical fibers were implanted at angles (DMS: 15°, 0.74 mm anterior, 1.1 mm lateral, -3.6 mm ventral; NAc: 10°, 1.3 mm anterior, 0.55 mm lateral, -5.0 ventral) and were then fixed to the skull using dental cement. Mice were allowed to recover and closely monitored for 5 days before beginning water-restriction and behavioral training.