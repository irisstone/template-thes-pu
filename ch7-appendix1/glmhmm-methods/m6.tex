\subsection{Conditioned place preference test}
\label{sec:ap1:m6}
Mice underwent a real-time conditioned place preference (CPP) test with bilateral optogenetic inhibition paired to one side of a two-chamber apparatus (Supplementary Fig. 3). The CPP apparatus consisted of a rectangular Plexiglass box with two chambers (29-cm x 25-cm) separated by a clear portal in the center. The same grey, plastic flooring was used for both chambers, but each chamber was distinguished by vertical or horizontal black and white bars on the chamber walls. During a baseline test, mice were placed in the central portal while connected to patch cables coupled to an optical commutator (Doric) and were allowed to freely move between both sides for 5 minutes. In a subsequent 20-min test, mice received continuous, bilateral optogenetic inhibition (532-nm, 5-mW) when located in one of the two chamber sides (balanced across groups). Video tracking, TTL triggering, and data analysis were carried out using Ethovision software (Noldus). Mice who displayed a bias for one chamber side greater than 45-s during the baseline test were excluded from analysis.