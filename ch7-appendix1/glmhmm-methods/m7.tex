\subsection{Behavior analyses}
\label{sec:ap1:m7}
\textit{Data selection}. For cross-task comparisons (Fig. 2-3; Extended Data Fig. 3-6) we analyzed only trials from evidence accumulation maze 10 (Supplementary Fig. 5a), “no distractors” maze 12 (Supplementary Fig. 5a,g), and “permanent cues” maze 8 (Supplementary Fig. 5g), which each followed matching rewarded- and non-rewarded side cue probability statistics (save for the by-design absence of non-rewarded cues in the “no distractors” control task). For model-based analyses of the evidence accumulation task (Fig. 4-7, Extended Data Fig. 7-10) both maze 10 and maze 11 data were included, which differed only modestly in the side ratio of reward to non-reward side cues (Supplementary Fig. 5a, ~50\% of trials were maze 10 or 11). In all tasks and all analyses throughout we removed initial warm-up blocks (Supplementary Fig. 5a, maze 4, approximately 5-15\% of total trials). For model-based analyses of the evidence accumulation task (Fig. 4-7, Extended Data Fig. 7-10), we included interspersed “easy blocks” capped at 10 trials in length (Supplementary Fig. 5a, maze 7, see description above). These trial blocks comprised approximately $\sim$5\% of total trials, were included to avoid gaps in trial history, and were treated identical to the main evidence accumulation mazes by the models. These trials were removed from cross-task comparisons of optogenetic inhibition (Fig. 2-3; Extended Data Fig. 3-6). \\
For analysis of optogenetic inhibition during virtual corridor navigation (Fig. 1 and Extended Data Fig. 2) we removed trials with excess travel $>$10\% of maze stem (or $>$330-cm) and mice with $<$150 total trials from measures of y-velocity, x-position, and average view angle. Trials with excess travel had similar proportions across laser off and laser on trials and pathway-specific inhibition and control groups (indirect pathway: 8.1\% of laser off and 8.2\% of laser on trials; direct pathway: 8.2\% of laser off and 8.1\% of laser on trials; no opsin control: 7.0\% of laser off and 6.9\% laser on trials; exact trial N in figure legends). Excess travel trials reflected the minority of trials in which mice made multiple traversals of the virtual corridor, thus skewing measures of average y-velocity, x-position, and view angle during the majority of “clean” corridor traversals. Importantly, we excluded no trials in direct measurements of distance, per-trial view angle standard deviation, and trials with excess travel in order to detect potential effects of pathway-specific DMS inhibition (or DMS illumination alone) on these measures (Fig. 1 and Extended Data Fig. 2). \\
Similarly, for all cross-task comparisons (Fig. 2-3; Extended Data Fig. 3-6) we removed trials with excess travel for all analyses comparing choice, y-velocity, x-position, and average view angle. To better capture task-engaged behavior we also only considered trial blocks in which choice accuracy was greater than $>$60\% for these measures. Excess travel trials were not excluded for cross-task comparisons of effects on measures of distance, per-trial view angle standard deviation, and trials with excess travel. Exact trial and mouse N are reported in figure legends throughout. \\
For cross-state comparisons of motor output measures (Extended Data Fig. 10) we did not exclude trial blocks with choice accuracy $<$60\%, given that GLM-HMM states were differentially associated with performance levels. However, for the reasons outlined above we removed excess travel trials from measures of y-velocity, x-position, and average view angle, and additionally only considered mice who occupied all three GLM-HMM states after trial selection. For measures of per-trial standard deviation in view angle, distance, and excess travel we applied no trial selection criteria, but only mice who occupied all three GLM-HMM states were included for analysis. Exact trial and mouse N are reported in the legend. \\
\textit{General performance indicators}. Accuracy was defined as the percentage of trials in which mice chose the maze arm corresponding to the side having the greater number of cues (Fig. 2c). For measures of choice bias, sensory evidence and choice were defined as either ipsilateral or contralateral relative to the unilaterally-coupled laser hemisphere. Choice bias was calculated separately for laser off and on trials as the difference in choice accuracy between trials where sensory evidence indicated a contralateral reward versus when sensory evidence indicated an ipsilateral reward (\% correct, contralateral-ipsilateral, positive values indicate greater contralateral choice bias) (Fig. 3d,g,j,o). Delta choice bias was calculated as the difference in contralateral choice bias between laser off and on trials (on-off, positive values indicate laser-induced contralateral choice bias) (Fig. 3e,g,k,p). In Extended Data Fig. 8c-f, reward at GLM-HMM state transition reflects the total amount of reward (mL) received from the start of the session up to the trial prior to a state transition (mice typically receive 1-1.5 mL per session). Reward rate at GLM-HMM state transition was calculated as the sum of reward received from the start of the session up to the trial prior to a state transition, divided by the sum of all trial durations from the start of the session up to the trial prior to a state transition. GLM-HMM transitions were defined as a within-session change in the most likely state based on the posterior probability of each state (see GLM-HMM below for more details). \\
\textit{Psychometric curve fitting}. Psychometric performance was assessed based on the percentage of contralateral choices as a function of the difference in the number of contralateral and ipsilateral cues (\#contra-\#ipsi) (Fig 6c-f; Extended Data Fig. 4). In Extended Data Fig. 4, transparent lines reflect the mean performance of individual mice in bins (-16:4:16, \#contra-\#ipsi) of sensory evidence during laser off (black) and on (green) trials, while bold lines reflect the corresponding mean and s.e.m across mice. In Fig. 6, psychometric curves were fit to the following 4-parameter sigmoid using maximum likelihood fitting \cite{wichmann_psychometric_2001}: 

\begin{equation} \label{meq3}
p\left( {{{{\mathrm{choice}}}} = R |{{\Delta }}} \right) = \lambda + \frac{{1 - \lambda - \gamma }}{{1 + {{{\mathrm{exp}}}}\left( { - \left( {{{\Delta }} - {\sigma}} \right){\mu}} \right)}}
\end{equation}

where $\lambda$ and $\gamma$ are the right and left lapse rates, respectively, $\sigma$ is the offset, $\mu$ is the slope, and $\Delta$ is the difference in the number of contralateral and ipsilateral cues on a given trial.  Each point in Fig. 6c-f represents the binned difference in cues in increments of 4 from -16 to 16 (as in Extended Data Fig. 4), from which we calculated the percentage of contralateral choice trials for each bin. \\
\textit{Motor performance indicators}. Y-velocity (cm/s) was calculated on every sampling iteration (120 Hz, or every ~8ms) of the ball motion sensor as dY/dt where dY was the change in Y-position displacement in VR and dt was the elapsed time from the previous sampling of the sensor. The y-velocity for all iterations in which a mouse occupied y-positions from 0-300-cm in 25-cm bins were then averaged across iterations in each bin to obtain per-trial y-velocity as a function of y-position. Binned y-velocity as a function of y-position was then averaged across trials for individual mice, and the average and standard error of the mean across mice reported throughout (Fig. 1g; Fig. 2d; Extended Data Fig. 6a-c; Extended Data Fig. 10b,p,i,w; averaged across y-position 0-200cm in Extended Data Fig. 2b and Extended Data Fig. 3b). \\
X-position trajectory (cm) as a function of y-position was calculated per trial by first taking the x-position at y-positions 0-300cm in 1-cm steps, which was defined as the x-position at the last sampling time t in which $y(t) \geq Y$, and then averaging across y-position bins of 25-cm from 0 to 300cm. Binned x-position as a function of y-position was then averaged across left/right (or ipsilateral/contralateral) choice trials for individual mice, and the average and standard error of the mean across mice was reported throughout (Fig. 1h; Fig. 2e; and Extended Data Fig. 10c,q; averaged across y-positions 0-200cm in Extended Data Fig. 2c; Extended Data Fig. 3c; Extended Data Fig. 6j-l; and Extended Data Fig. 10j,x). Average view angle trajectory (degrees) was calculated in the same manner as x-position (Fig. 1i; Fig. 2f; and Extended Data Fig. 10d,r; average across y-positions 0-200cm in Extended Data Fig. 2d; Extended Data Fig. 3d; Extended Data Fig. 6j-l; and Extended Data Fig. 10k,y). View angle standard deviation was calculated by first sampling the per-trial view angle from 0-300cm of the maze in 5-cm steps. The standard deviation in view angle was then calculated for each trial, and then averaged across trials for individual mice. The average and standard error of the mean across mice are reported throughout (Extended Data Fig. 2e; Extended Data Fig. 3f; Extended Data Fig. 6g-i; and Extended Data Fig. 10e,s,l,z). This measure sought to capture unusually large deviations in single trial view angles, which would be indicative of excessive turning or rotations. \\
Distance traveled was measured per trial as the sum of the total x and y displacement calculated at each sampling iteration t, as $\sqrt {dX^2 + dY^2}$ . Distance traveled per trial was then averaged across trials for individual mice and the average and standard error of the mean across mice was reported throughout (Fig. 1j; Fig. 2g; Extended Data Fig. 2f; Extended Data Fig. 6d-f, left; Extended Data Fig. 10f,t,m,aa). Excess travel was defined as the fraction of trials with total distance traveled per trial (calculated as above) greater than 10\% of maze length (or $>$330cm). The average and standard error of the mean across mice was reported throughout (Extended Data Fig. 2g; Extended Data Fig. 3e; Extended Data Fig. 6d-f, right; and Extended Data Fig. 10g,u,n,bb). \\
Decoding of choice based on the trial-by-trial x-position (Extended Data Fig. 3g,i) or view angle (Extended Data Fig. 3h,j) of mice was carried out by performing a binomial logistic regression using the MATLAB function glmfit. In Extended Data Fig. 3g-h, the logistic regression was fit separately for individual mice at successive y-positions in each T-maze stem (0-300cm in 25-cm bins), where the trial-by-trial average x-position (or view angle) at each y-position bin (calculated as above) was used to generate weights predicting the probability of a left or right choice given a particular x-position (or view angle) value. Individual mouse fits were weighted according to the proportion of left and right choice trials. 5-fold cross-validation (re-sampled for new folds 10 times) was used to evaluate prediction accuracy on held-out trials. A choice probability greater than or equal to 0.5 was decoded as a right choice, and prediction accuracy for individual mice was calculated as the fraction of predicted choices matching actual mouse choice, averaged across cross-validation sets. The same approach was used in Extended Data Fig. 3i-j, except that the training data was randomly sampled across all mice from a single task (50\% of total trials, re-sampled 50 times; train on data from evidence accumulation, no distractors, or permanent cues task). The learned weights were then used to predict choice based on held out x-position (or view angle) data from all three tasks, with prediction accuracy calculated as the fraction of predicted choices matching actual choice, parsed by individual mice, and averaged across cross-validation sets. A package of code for behavioral analysis in VR-based T-maze settings is available at: https://github.com/BrainCOGS/ behavioralAnalysis. In addition, all analyses described here can be replicated at https://github.com/ssbolkan/BolkanStoneEtAl.