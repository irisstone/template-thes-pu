\subsection{Optrode recording for NpHR validation}
\label{sec:ap1:m3}
Following the surgical procedures described above, Cre-dependent NpHR was virally delivered bilaterally to the DMS of mice (n = 3 A2a-Cre; n = 2 D1R-Cre) via small (~300-uM) craniotomies made using a carbide drill (Extended Data Fig. 1a). The craniotomies were filled with a small amount of silicon adhesive (Kwik-Sil, World Precision instruments) and then covered with UV-curing optical adhesive (Norland Optical Adhesive 61), while a custom-designed headplate for head-fixation was cemented to the skull. Following a recovery period of >4 weeks, awake mice were head-fixed on a plastic running wheel attached to a breadboard via Thorlabs posts and holders, which was fixed immediately adjacent to a stereotaxic setup (Kopf) enclosed within a Faraday cage (Extended Data Fig. 1b). Silicon and optical adhesive was removed from the craniotomies and a 32-channel, single-shank silicon probe (A1x32-Poly3, NeuroNexus) coupled to a tapered optical fiber (65 uM, 0.22 NA) was stereotaxically inserted under visual guidance of a stereoscope and allowed to stabilize for ~30 minutes. Signals were acquired at 20 kHz using a digital headstage amplifier (RHD2132, Intan) connected to an RHD USB data acquisition board (C3100, Intan). A screw implanted over the cerebellum served as ground. Continuous signal was imported into MATLAB for referencing to a local probe channel and high-pass filtering at 200 Hz, and then imported into Offline Sorter v3 (Plexon) for spike thresholding and single-unit sorting. During recording, the optical fiber was connected via a patch cable to a 532-nm laser, which was triggered by a TTL pulse sent by a pulse generator controlled by a computer running Spike2 software. TTL pulse times were copied directly to the RHD USB data acquisition board. Laser sweeps consisted of forty deliveries of 5-s light (5-mW, measured from fiber tip), separated by 15-s intertrial intervals. From 1-3 recordings were made at different depths within a single probe penetration (minimum separation of 300-uM), with each hemisphere receiving 1-3 penetrations at different medial-lateral or anterior-posterior coordinates. For recordings in mice carried out over multiple days, craniotomies were filled with Kwik-Sil and covered with silicone elastomer between recordings (Kwik-Cast, World Precision Instruments).