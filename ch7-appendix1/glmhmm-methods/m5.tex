\subsection{Optogenetics during VR behavior}
\label{sec:ap1:m5}
According to a previously published protocol \cite{hanks_distinct_2015}, optical fibers (200uM, 0.37 NA) were chemically etched using 48\% hydrofluoric acid to achieve tapered tips of lengths 1.5-2 mm (DMS-targeted) or 1-1.5 mm (NAc-targeted). Following behavioral shaping in VR (and >6weeks of viral expression) mice underwent optogenetic testing. On alternate daily sessions, optical fibers in the left or right hemisphere were unilaterally coupled to a 532-nm DPSS laser (Shanghai, 200 mW) via a multi-mode fiberoptic patch cable (PFP, 62.5 µM). On a random subset of trials (10-30\%), mice received unilateral laser illumination (5 mW, measured from patch cable) that was restricted to the first passage through 0-200-cm of the virtual corridor (Fig. 1 and Extended Data Fig. 2), or the cue region (0-200cm) of each T-maze decision-making task (Fig. 3). The laser was controlled by TTL pulses generated using a National Instruments DAQ card on a computer running the ViRMEn-based virtual environment. 


