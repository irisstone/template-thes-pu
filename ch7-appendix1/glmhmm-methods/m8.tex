\subsection{General statistics and Reproducibility}
\label{sec:ap1:m8}
We performed one-way ANOVAs to assess effects of the factor task (three levels: evidence accumulation, no distractors, or permanent cue) on choice accuracy (Fig. 2c), distance traveled (Fig. 1g), average y-velocity (0-200cm) (Extended Data Fig. 3b), average x-position (0-200cm) on \textit{left} or \textit{right} choice trials (Extended Data Fig 6c), average view angle (0-200cm) on \textit{left} or \textit{right} choice trials (Extended Data Fig. 3d), percent trials with excess travel (Extended Data Fig. 3e), per-trial standard deviation in view angle (Extended Data Fig. 3f), delta (laser on-off) choice bias (Extended Data Fig. 5b-d), delta (laser on-off) distance traveled (Extended Data Fig. 6d-f, \textit{left}), delta (laser on-off) percent trials with excess travel (Extended Data Fig. 6d-f, \textit{right}), delta (laser on-off) per-trial standard deviation in view angle (Extended Data Fig. 6g-i), delta (laser on-off) average x-position (0-200cm) (Extended Data Fig. 6j-l, \textit{left}), and delta (laser on-off) average view angle (0-200cm) (Extended Data Fig. 6j-l, \textit{right}). Post-hoc comparisons between tasks were made when a main effect of the factor task had a p-value less than alpha <0.05/2 to account for multiple group comparisons (Extended Data Fig. 5b-d). One exception to this is Extended Data Fig. 6j-l, where all post-hoc comparisons were made for laser effects on delta (on-off) x-position and view angle (and displayed with corresponding exact p-values) to provide greater clarity around trend-level effects. We did not assume normality of the data in all post-hoc comparisons, which used the non-parametric, unpaired, two-tailed Wilcoxon rank sum test. A p-value below 0.025 was considered significant in order to correct for multiple comparisons (p = 0.05/2 comparisons per group). Exact p-values, degrees of freedom, and z-statistics are reported in the text and/or legends. \\
We performed one-way ANOVAs to assess effects of the factor group (three levels: indirect pathway inhibition, direct pathway inhibition, or no opsin illumination) on delta (laser on-off) y-velocity (Extended Data Fig. 2b), delta (laser on-off) x-position (Extended Data Fig. 2c), delta (laser on-off) view angle (Extended Data Fig. 2d), delta (laser on-off) per-trial standard deviation in view angle (Extended Data Fig. 2e), delta (laser on-off) distance (Extended Data Fig. 2f), delta (laser on-off), and delta (laser on-off) percent trials with excess travel (Extended Data Fig. 2g), and delta (laser on-off) preference (i.e. time) and speed during the real-time conditioned place preference test (Supplementary Fig. 3). \\
We performed a repeated-measure one-way ANOVA to assess effects of the within-subject factor state (three levels: GLM-HMM state 1, 2, or 3) on within-session accumulated reward or reward rate prior to GLM-HMM transition (Extended Data Fig. 8c-f), per-trial standard deviation in view angle (Extended Data Fig. 10e,s), distance traveled (Extended Data Fig. 10f,t), percent trials with excess travel (Extended Data Fig. 10g,u), delta (laser on-off) average x-position 0-200cm (Extended Data Fig. 10j,x), delta (laser on-off) average view angle 0-200cm (Extended Data Fig. 10k,y), delta (laser on-off) per-trial standard deviation in view angle (Extended Data Fig. 10l,z), delta (laser on-off) distance traveled (Extended Data Fig. 10m,aa), and delta (laser on-off) percent trials with excess travel (Extended Data Fig. 10n,bb). Post-hoc comparisons between groups were made when a main effect of the factor task had a p-value <0.05. One exception is in Extended Data Fig. 6j,k,x,y, where all post-hoc comparisons were made for laser effects on delta (on-off) x-position and view angle (and displayed with corresponding exact p-values) to provide greater clarity around trend-level effects. We did not assume normality of the data in all post-hoc comparisons, which used the non-parametric, unpaired, two-tailed Wilcoxon rank sum test. A p-value below 0.025 was considered significant in order to correct for multiple comparisons (p = 0.05/2 comparisons per group). Exact p-values, degrees of freedom, and z-statistics are reported in the text and/or legends. \\
In Fig. 3e,h,k,p, we used the non-parametric, unpaired, two-tailed Wilcoxon rank sum test to assess effects of indirect or direct pathway inhibition versus no opsin illumination of brain tissue on delta (laser on-off) choice bias. A p-value below 0.025 was considered significant in order to correct for multiple comparisons (p = 0.05/2 comparisons per group). Exact p-values, degrees of freedom, and z-statistics are reported in the text and/or legends. \\
Related to Fig. 1a-c, Extended Data Fig. 1, and Supplementary Fig. 2, we used a paired, two-tailed Wilcoxon signrank test on cross-trial average firing rates (baseline pre versus laser on or laser on versus laser post) to determine significance of laser modulation of single neuron activity. A bonferroni-corrected p-value was used to determine significance (p < 0.00083 for 60 indirect pathway neurons or p < 0.001 for 50 direct pathway neurons). \\
All experiments, but not analysis, were conducted blind to experimental conditions. All experiments were conducted over multiple cohorts. Individual cohorts consisted of a random selection of test groups (e.g. DMS indirect and direct pathway inhibition, and DMS no-opsin mice). We did not account for cohort effects in our statistical analyses, but no obvious cohort-dependent effects were qualitatively observable. No statistical method was used to predetermine group sample sizes; rather animal and trial N were targeted to match or exceed similar studies.

