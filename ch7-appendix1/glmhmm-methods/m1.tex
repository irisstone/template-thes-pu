\subsection{Animals}
\label{sec:ap1:m1}
For optogenetic experiments we used both male and female transgenic mice on heterozygous backgrounds, aged 2-6 months of age, from the following three strains backcrossed to a C57BL/6J background (Jackson Laboratory, 000664) and maintained in-house: Drd1-Cre (n = 45, EY262Gsat, MMRRC-UCD), Drd2-Cre (n = 24, ER44Gsat, MMRRC-UCD), and A2a-Cre (n = 18, KG139Gsat, MMRRC-UCD). An additional 35 mice were excluded from all optogenetic analyses due to failed task acquisition (n = 11 mice) or failed viral/fiberoptic targeting of DMS (n = 8) or NAc (n = 16). An additional 4 Drd1-Cre mice, 3 A2a-Cre mice, and 2 Drd2-Cre mice were used for electrophysiological characterization of halorhodopsin (NpHR)-mediated inhibition, or fluorescent in situ hybridization (FISH) characterization of Cre expression profiles. FISH experiments also utilized 2 Drd1a-tdTomato mice (Jax, 016204). Mice were co-housed with same-sex littermates and maintained on a 12-hour light – 12-hour dark cycle. All surgical procedures and behavioral training occurred in the dark cycle. All procedures were conducted in accordance with National Institute of Health guidelines and were reviewed and approved by the Institutional Animal Care and Use Committee at Princeton University.