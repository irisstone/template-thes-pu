\subsection{Fluorescence in situ hybridization and stereological quantification}
\label{sec:ap1:m11}
In situ hybridization (Supplementary Fig. 1) was performed using the RNAscope Multiplex Fluorescent Assay (ACD, no. 323110) with the following probes: Mm-Drd1a (406491), Mm-Drd2-C2 (406501-C2, 1:50 dilution in C1 solution) and Cre-01-C3 (474001-C3, 1:50 dilution in C1 solution). Likely due to lower expression of Cre mRNA in D1R-Cre and A2a-Cre mice, we did not detect unambiguous Cre fluorescence signal in these lines. We therefore relied on Cre-dependent viral expression of AAV5-DIO-EYFP to report Cre+ neurons alongside Drd1a and Drd2 probes in sections from two D1R-Cre and two A2R-Cre mice, but used all three probes in sections from two D2R-Cre mice. In D1R-Cre and A2R-Cre mice, the Drd1a and Drd2 probes were fluorescently linked to TSA Plus Cy-3 and TSA Plus Cy-5, respectively (Perkin Elmer). In D2R-cre mice, Drd1a, Drd2 and Cre probes were linked to TSA Plus Cy-3, TSA Plus Fluorescein or TSA Plus Cy-5, respectively. All fluorophores were reconstituted in dimethylsulfoxide according to Perkin Elmer instructions and diluted 1:1,200 in TSA buffer included in the RNAscope kit. After in situ hybridization, slides were coverslipped using Fluoromount-G containing DAPI (SouthernBiotech). \\
We then obtained 20x confocal z-stacks from the DMS, NAcCore, and NAcShell in all lines and manually quantified specificity, penetrance, and D1R+/D2R+ overlap using LASX software (Leica). Specificity was determined as the percentage of the following: GFP+ neurons co-expressing Drd1 in D1R-Cre mice (DMS, n = 5 sections, 193 cells; NAcCore, n = 5 sections, 298 cells; NAcShell, n = 5 sections, 363 cells), GFP+ neurons co-expressing Drd2 in A2A-Cre mice (DMS, n = 4 sections, 144 cells; NAcCore, n = 4 sections, 326 cells; NAcShell, n = 4 sections, 312 cells), or Cre+ neurons co-expressing Drd2 in D2R-Cre mice (DMS, n = 5 sections, 1,302 cells; NAcCore, n = 5 sections, 1,104 cells; NAcShell, n = 5 sections, 1,187 cells). Penetrance was determined as the percentage of Drd2+ neurons co-expressing Cre in D2R-Cre mice (DMS, n = 5 sections, 1,269 cells; NAcCore, n = 5 sections, 1,055 cells; NAcShell, n = 5 sections, 1,144 cells). We did not assess penetrance in D1R-Cre or A2a-Cre lines because our Cre-dependent viral reporter did not fully penetrate all Cre+ neurons. Quantification of D1R+/D2R+ overlap in striatal regions was carried out on 2 D2R-Cre mice and/or 2 D1R-tdTomato mice and measured as both the percentage of D1R+ neurons that were D2R+ (DMS, n = 10 sections, 2,423 cells; NAcCore, n = 10 sections, 2,196 cells; NAcShell, n = 10 sections, 2,220 cells) and the percentage of D2R+ neurons that were D1R+ (DMS, n = 5 sections, 868 cells; NAcCore, n = 5 sections, 834 cells; NAcShell, n = 5 sections, 874 cells).
