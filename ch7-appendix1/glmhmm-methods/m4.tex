\subsection{VR Behavior}
\label{sec:ap1:m4}
\textit{Virtual reality setup}. Mice were head-fixed over an 8-inch Styrofoam® ball suspended by compressed air (~60 p.s.i.) facing a custom-built Styrofoam® toroidal screen spanning a visual field of 270° horizontally and 80° vertically. The setup was enclosed within a custom-designed cabinet built from optical rails (Thorlabs) and lined with sound-attenuating foam sheeting (McMaster-Carr). A DLP projector (Optoma HD141X) with a refresh rate of 120 Hz projected the VR environment onto the toroidal screen (Fig. 1e). \\
An optical flow sensor (ADNS-3080 APM2.6), located beneath the ball and connected to an Arduino Due, ran custom code to transform real-world ball rotations into virtual-world movements (https://github.com/sakoay/AccumTowersTools/tree/ master/OpticalSensorPackage) within the Matlab-based ViRMEn54 software engine (http://pni.princeton.edu/pni-software-tools /virmen). The ball and sensor of each VR rig were calibrated such that ball displacements (dX and dY, where X (and Y) are parallel to the anterior-posterior (and medial-lateral) axes of the mouse) produced translational displacements proportional to ball circumference in the virtual environment of equal distance in corresponding X and Y axes. The y-velocity of the mouse was given by $\sqrt {dY^2/dt}$, where dt was the elapsed time from the previous sampling of the sensor. The virtual view angle of mice was obtained by first calculating the current displacement angle as: $\omega = atan2\left( { - dX \times sin(dY),\;\left| {dY} \right|} \right)$. Then the rate of change of view angle ($\theta$) for each sampling of the sensor was given by:

\begin{equation} \label{meq1}
\frac{{d\theta }}{{dt}} = \sin \left( \omega \right) \times \min \left( {e^{1.4\left| \omega \right|^{1.2}} - 1,\;\frac{\pi }{2}} \right) - \theta
\end{equation}

This exponential function was tuned to (1) minimize the influence of small ball displacements and thus stabilize virtual-world trajectories, and (2) increase the influence of large ball displacements in order to allow sharp turns into the maze arms \cite{pinto_accumulation--evidence_2018}. \\

Reward and whisker air puffs were delivered by sending a TTL pulse to solenoid valves (NResearch), which were generated according to behavioral events on the ViRMEn computer. Each TTL pulse resulted in either the release of a drop of reward (~4–8µl of 10\% sweetened condensed milk in water, vol/vol) to a lick tube, or the release of air flow (40 ms, 15 p.s.i.) to an air puff cannula (Ziggy’s Tubes and Wires, 16 gauge) directed to the left and right whisker pads from the rear position. The ViRMEn computer also controlled TTL pulses sent directly to a 532-nm DPSS laser (Shanghai, 200 mW).

\textit{Behavioral shaping.} Following post-surgical recovery, over the course of 4–7 d, mice were extensively handled while gradually restricting water intake to an allotted volume of 1–2 ml per day. Throughout water restriction, mice were closely monitored to ensure no signs of dehydration were present and that body mass was at least 80\% of the pre-restriction value. Mice were then introduced to the VR setup where behavior was shaped to perform the accumulation of evidence task as previously described in detail25 (Supplementary Fig. 5a) or the permanent cues (control 2) task (Supplementary Fig. 5f). We tested a total of 32, 34 and 20 mice in the accumulation of evidence, no distractors (control 1) and permanent cues (control 2) tasks, respectively. No mice received optogenetic testing in all three tasks, but 7 mice received optogenetic testing in the accumulation of evidence and no distractors task, and 19 mice received optogenetic testing in the no distractors and permanent cues tasks.\\
Shaping followed a similar progression in both tasks. In the first four shaping mazes of both procedures, a visual guide located in the rewarded arm was continuously visible, and the maze stem was gradually extended to a final length of 300-cm (Supplementary Fig. 5a,f). In mazes 5-7 of the evidence accumulation shaping procedure (Supplementary Fig. 5a), the visual guide was removed and the cue region was gradually decreased to 200-cm, thus introducing the full 100-cm delay region of the testing mazes. The same shift to a 200-cm cue region and 100-cm delay region occurred in mazes 5-6 of the permanent cues shaping procedure, but without removing the visibility of the visual guide (Supplementary Fig. 5f). In mazes 8-9 of evidence accumulation shaping, distractor cues were introduced to the non-rewarded maze side with increasing frequency (mean side ratio (s.d.) of rewarded::non-rewarded side cues of 8.3::0.7 to 8.0::1.6 m-1). Distractor cues were similarly introduced with increasing frequency in mazes 6-8 of the permanent cues shaping procedure, while the visual guide was removed in maze 7 and 8. In all evidence accumulation shaping mazes (maze 1-9) cues were only made visible when mice were 10-cm from the cue location and remained visible until trial completion. In the final evidence accumulation testing mazes (maze 10 and 11) cues were made transiently visible (200-ms) after first presentation (10-cm from cue location), while the mean side ratio of rewarded::non-rewarded side cues changed from 8.0::1.6 (Supplementary Fig. 5a, maze 10) to 7.7::2.3 m-1 (Supplementary Fig. 5a, maze 11). In contrast, throughout all shaping (maze 1-6) and testing mazes (maze 7-8) of the permanent cues task, cues were visible from the onset of a trial. \\
The median number of sessions to reach the first evidence accumulation testing maze (maze 10) was 22 sessions, while the mean number of sessions was 23.0 $\pm$ 0.8 (Supplementary Fig. 5b-c). Mice typically spent between 2-5 sessions on each shaping maze before progressing to the next, with performance increasing or remaining stable throughout (Supplementary Fig. 5d-e; maze 9: 74.1 $\pm$ 9.8 percent correct). The median number of sessions to reach the first permanent cues (control \#2) testing maze (maze 7) was 17 sessions, while the mean number of sessions was 18.0 $\pm$ 1.5 (Supplementary Fig. 5g-j). Mice typically spent between 2-4 sessions on each shaping maze before progressing to the next, with performance increasing or remaining largely stable throughout (Supplementary Fig. 5g-j; maze 6: 87.0 $\pm$ 4.3 percent correct). \\
\textit{Optogenetic testing mazes}. The evidence accumulation task took place in a 330-cm long virtual T-maze with a 30-cm start region (-30 to 0-cm), followed by a 200-cm cue region and finally a 100-cm delay region (Fig. 2a, black, left). While navigating the cue region of the maze mice were transiently presented with high-contrast visual cues (wall-sized “towers”) on either maze side, which were also paired with a mild air puff (15 p.s.i, 40-ms) to the corresponding whisker pad. The side containing the greater number of cues indicated the future rewarded arm. A left or right choice was determined when mice crossed an x-position threshold $> |15cm|$, which was only possible within one of the maze arms (the width of choice arms were $\pm$ 25-cm relative to the center of the maze stem). Mice received reward (~4-8 µL of 10\% v/v sweetened condensed milk in drinking water) followed by a 3-s ITI after turning to the correct arm at the end of the maze, while incorrect choices were indicated by a tone followed by a 12-s ITI. In each trial, the position of cues was drawn randomly from a spatial Poisson process with a rate of 8.0 m-1 for the rewarded side and 1.6 m-1 for the non-rewarded side (Supplementary Fig. 5a, maze 10) or 7.3::2.3 m-1 (Supplementary Fig. 5a, maze 11). Note that only maze 10 data was used for cross-task comparisons of optogenetic effects with permanent cues and no distractors control tasks in order to precisely match cue presentation statistics (Fig. 2-3; Extended Data Fig. 3-6). Visual cues (and air puffs) were presented when mice were 10-cm away from their drawn location and ended 200-ms (or 40-ms) later. Cue positions on the same side were also constrained by a 12-cm refractory period. Each session began with warm-up trials of a visually-guided maze (Supplementary Fig. 5a, maze 4), with mice progressing to the evidence accumulation testing maze after 10 trials (or until accuracy reached 85\% correct). During performance of the testing maze if accuracy fell below 55\% over a 40-trial running window, mice were transitioned to an easier maze in which cues were presented only on the rewarded side and did not disappear following presentation (Supplementary Fig. 5a, maze 7). These “easy blocks” were limited to 10 trials, after which mice returned to the main testing maze regardless of performance. Behavioral sessions lasted for $\sim$1-hour and typically consisted of $\sim$150-200 trials. \\
All features of the “no distractors” (control \#1) task (Fig. 2b, magenta, middle; Supplementary Fig. 5a,g, maze 12) were identical to the evidence accumulation task (Supplementary Fig. 5a, maze 10) except: (1) distractor cues were removed from the non-rewarded side, and (2) a distal visual guide located in the rewarded arm was transiently visible during the cue region (0-200-cm). \\
All features of the “permanent cues” (control \#2) task (Fig. 2b, cyan, right; Supplementary Fig. 5g, maze 8) were identical to the evidence accumulation task except: (1) reward and non-reward side visual cues were made permanently visible from trial onset. As in the evidence accumulation task, whisker air puffs were only delivered when mouse position was 10-cm from visual cue location. Note that mice underwent optogenetic testing on two permanent cues mazes (maze 7 and maze 8). Maze 8 shared identical reward to non-reward side cue statistics (8.0::1.6 m-1) as maze 10 of the evidence accumulation task. Therefore, for all cross-task comparisons of optogenetic inhibition only data from these mazes were analyzed (Fig. 2-3; Extended Data Fig. 3-6). \\
To discourage side biases, in all tasks we used a previously implemented debiasing algorithm \cite{pinto_accumulation--evidence_2018}. This was achieved by changing the underlying probability of drawing a left or a right trial according to a balanced method described in detail elsewhere\cite{pinto_accumulation--evidence_2018}. In brief, the probability of drawing a right trial, $p_R$, is given by:

\begin{equation} \label{meq2}
p_R = \frac{{\sqrt {e_R} }}{{\left( {\sqrt {e_R} + \sqrt {e_L} } \right)}}
\end{equation}

Where $e_R$ (and $e_L$) are the weighted average of the fraction of errors the mouse has made in the past 40 right (and left) trials. The weighting for this average is based on a half-Gaussian with $ \sigma = 20$ trials in the past, which ensures that most recent trials have larger weight on the debiasing algorithm. To discourage the generation of sequences of all-right (or all-left) trials, we capped $\sqrt{e_R}$ and $\sqrt{e_L}$ to be within the range $[0.15, 0.85]$. Because the empirical fraction of drawn right trials could significantly deviate from $p_R$, particularly when the number of trials is small, we applied an additional pseudo-random drawing prescription to $p_R$. Specifically, if the empirical fraction of right trials (calculated using a $ \sigma = 60$ trials half-Gaussian weighting window) is above $p_R$, right trials were drawn with probability $0.5p_R$, whereas if this fraction is below $p_R$, right trials were drawn with probability $0.5(1+p_R)$. \\
\textit{Virtual corridor}. Following shaping in the behavioral tasks above mice were transitioned to free navigation in a virtual corridor arena in the same VR apparatus described above. The virtual corridor was 6-cm in diameter and 330-cm in effective length (Fig. 1e-f). This included a start region (-10 to 0-cm), a reward location (310-cm) in which mice received 4 µL of 10\% v/v sweetened condensed milk in drinking water, and a teleportation region (320-cm) in which mice were transported back to the start region following a variable ITI with mean of 2-s. Mice were otherwise allowed to freely navigate the virtual corridor over the course of ~70 minute sessions. The virtual environment was controlled by the ViRMEn software engine, with real-to-virtual world movement transformations as described above.