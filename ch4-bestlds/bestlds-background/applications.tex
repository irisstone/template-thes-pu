\subsection{Applications of state-space models in neuroscience}
\label{sec:bestlds:background:applications}

LDS and related state-space models have a long history in neuroscience. Bernoulli- and Poisson-LDS models have been particularly popular for inferring latent processes underlying neural spike trains \cite{gao_high-dimensional_2015, nonnenmacher_extracting_2017, zoltowski_general_2020, valente_probing_2022}. LDS models are also useful for linking neural dynamics to animal behavior \cite{linderman_hierarchical_2019}, while discrete state-space models like hidden Markov models (HMMs) have become increasingly common tools for characterizing behavior directly \cite{wiltschko_mapping_2015, calhoun_unsupervised_2019, wiltschko_revealing_2020}. More recently, there has been growing interest in using state-space models to describe behavior: (1) using  continuous rather than discrete variables \cite{johnson_composing_2016, costacurta_distinguishing_2022}; (2) specifically in two-alternative forced-choice (2AFC) decision-making contexts \cite{roy_extracting_2021};  and (3) while understanding the affects of inputs such as sensory stimuli \cite{calhoun_unsupervised_2019, bolkan_opponent_2022, ashwood_mice_2022}. Bernoulli-LDS models lie at the intersection of all three objectives, and thus their utility (along with methods that make their inference more efficient) will likely be central to neuroscience research in the future.