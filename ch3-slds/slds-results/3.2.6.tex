\subsection{Subspace identification recovers LDS parameters and provides good EM initializations in simulated data}
\label{sec:slds:3.2.6}

Although SSID performance has been well-documented in Gaussian LDS models, natural behavior data is particularly complex and high-dimensional. Therefore, we started by assessing the ability of SSID to return accurate parameters when fitting LDS models to simulations of natural behavior data. To generate simulated datasets, we first fit an LDS to the subset of the real data associated with each behavior. In order to have sufficient real data for each behavior ($\sim300k$ frames) to be sufficiently confident that fitting would obtain the global optimum, we used a larger real dataset than that described in Section \ref{sec:slds:3.2.1} containing videos in other stimulus conditions (aversive odor and food reward) in addition to the empty condition. We then used the model-inferred system parameters as the true parameters for simulated datasets in which the keypoint temporal dynamics matched what one would expect to see in sitting, walking, and rearing behaviors. To investigate how the dimensionality of the continuous latents affected SSID performance, we repeated this process for LDS models with 3, 15, and 26 continuous latents (where we considered $p=3$ the minimum dimensionality needed to capture basic pose dynamics and  $p=26=q$). 