\section{Discussion}
\label{sec:slds:discussion}

Behavioral segmentation and quantification is a topic of keen interest to the neuroscience community for its potential to precisely and reliably map observable actions to underlying representations of neural activity. The best computational methods for achieving this goal is an open question and active area of exploration amongst researchers, and the answer may depend on behavioral context, data characteristics, and questions of interest. Latent variable models are one class of methods that have become particularly popular because their underlying statistical structure reflects the exact brain-behavior relationship that neuroscientists hope to capture: a time series of observable variables (behavior) governed by an underlying dynamic process that may evolve in a continuous and/or discrete manner over time and is hidden from direct observation (brain activity). However, the full utility of these models and their generalizability across contexts, data, and questions remains to be seen.  

Here we have presented two latent variable models for quantifying mammalian behavior and compared their performance in identifying basic behavior states as mice explore a linear track. The first, an AR-HMM, is the more familiar approach in the literature \cite{wiltschko_mapping_2015, markowitz_striatum_2018, wiltschko_revealing_2020, costacurta_distinguishing_2022}. The second, an SLDS, can be seen as an extended version of an AR-HMM with an additional state layer that represents a denoised projection of the low-level observations. We compared the two models across both quantifiable and qualitative metrics to assess measures such as their accuracy in identifying ground-truth behaviors as well as their ability to capture states that reflect interpretable, distinct types of actions. 

Our findings indicate that the SLDS performs better than the AR-HMM across most measures. Average discrete state durations are longer and self-transition probabilities higher, especially for stationary behaviors. This is likely due to the denoising component of the SLDS, as it enables capture of long periods when the mouse is not moving without being artificially disrupted by noise in the keypoint positions. States are also more distinguishable from one another in terms of the animals' average pose and sideways and upwards velocities in each state. Mapping these states back onto the video recordings, the SLDS states can be more readily associated with different behaviors (i.e. stationarity and combinations of movements at differing speeds) than the AR-HMM inferred states, which don't clearly cluster into any distinct behaviors. 

Direct model comparisons further support the superiority of the SLDS model. The SLDS performs better at predicting the true states (compared to ground truth labels) on test data and reports lower errors than the AR-HMM. It also does a much better job at classifying stationary behavior, though it does somewhat worse at distinguishing between two different movement behaviors (rearing and walking). 

While the results show high potential for the use of SLDS models in behavioral quantification, they also reveal several areas of improvement. The average state durations are still much shorter than what is typically associated with discrete behaviors \cite{wiltschko_mapping_2015}. This may speak to the need to refine the magnitude of the latent noise term in order to further reduce jitter in the observations or to impose a dynamic noise term that varies over time in accordance with the amount of observed noise in the keypoints. One recent attempt to implement the latter has garnered positive results, with average state durations approaching 400ms \cite{weinreb_keypoint-moseq_2023}. 

Another limitation in our SLDS model is that the behaviors associated with the inferred non-stationary states are neither sufficiently well distinguished nor score well in their precision and recall of ground truth behaviors. This problem is likely solveable by adding a velocity term as an input to the model (possibly with separate velocity terms for different axes) to help distinguish the pose trajectories associated with different behaviors. Early evidence suggests that including at least a centroid velocity term, along with the animal's heading direction in order to capture overall position in allocentric coordinates, helps to distinguish between different types of locomotion such as walking, rearing, and turning \cite{weinreb_keypoint-moseq_2023}. 

There are of course additional considerations to optimize SLDS performance, including adding discrete states to uncover more behaviors and using cross-validation or singular value decompositions of the data to determine the appropriate dimensionality of the continuous states. It is likely that the model would produce equivalent or even superior results with a continuous latent space that is of reduced dimension relative to the observation space, preserving only the most relevant information to infer the discrete state behavioral dynamics. Reducing the number of continuous latent dimensions can also vastly speed up inference, which is a substantial problem in fitting SLDS models. 

Another factor that influences fitting efficiency is the parameter scheme used to initialize model inference. One approach is to generate initial parameters using a combination of factor analysis and/or AR-HMMs to obtain estimates of the continuous and discrete state dynamics \cite{linderman_recurrent_2016, linderman_hierarchical_2019, weinreb_keypoint-moseq_2023}. However, fitting AR-HMMs to keypoint data does not produce state sequences reflective of the data, as has been shown here and elsewhere \cite{wu_deep_2020, luxem_identifying_2022, weinreb_keypoint-moseq_2023}. Therefore, alternative approaches to SLDS initialization, such as the fastSLDS method we propose here, will be important to explore more thoroughly as wide spread adoption of SLDS models for behavioral quantification becomes likely.  

Altogether, this work provides an important examination of two latent variable models for behavioral quantification and raises critical questions about the right computational tools for extracting relevant information from behavioral data. We also suggest essential considerations about the efficiency of fitting such models and propose a novel solution to reduce inference time. To this end, we expect our work to provide a useful foundation for future studies to expand upon as the field pinpoints the optimal techniques for modeling behavioral data and linking its dynamic structures to underlying patterns of neural activity. 