\chapter{Conclusion and Outlook\label{ch:conclusion}}

In this thesis, we have introduced a number of different latent variable models for characterizing the dynamic structure underlying complex behaviors. Each method we covered aimed to address a current gap in the field and push the bounds of neuroethology towards a more enriched understanding of mammalian behavior. For example, sought insights into the external (e.g. sensory) and internal factors that guide behavior as well as the neural correlates of those behaviors. We also sought to produce technical innovations to models of behavior as a way to improve their performance and interpretability as well as their computational efficiency.

In Chapter \ref{ch:glmhmm}, we introduced the ``GLM-HMM" as a flexible framework for understanding binary choice behavior in a cognitively-demanding decision-making task wherein mice had to accumulate sensory evidence while navigating a virtual maze.  The GLM-HMM revealed that mice switch between multiple decision-making strategies when performing the task, and that these states had differing dependence on the DMS, a brain region associated with decision-making \cite{balleine_role_2007, ding_separate_2012, yartsev_causal_2018, cox_striatal_2019} that received inhibition on a subset of task trials. This result raises novel questions about the neural circuitry underlying decision-making and other higher-order cognitive processes. Is it possible that the neural correlates of cognition more generally are not static, but rather dynamic according to internal changes in an animal's state? How might these results generalize or change across species? Recent work suggests that humans also use different strategies during decision-making \cite{ashwood_mice_2022}, although it remains to be seen if these states may correspond to differences in neural activity. 

In Chapter \ref{ch:slds}, we asked whether models like the GLM-HMM can also be used to describe unconstrained ``natural" behaviors such as exploration. While previous work had used a version of an input-output HMM called an ``AR-HMM" to model high-dimensional natural behaviors, these studies were limited to situations in which the data was acquired from depth imaging cameras, which are smoothly-varying over time but lack spatial resolution \cite{wiltschko_mapping_2015, markowitz_striatum_2018, wiltschko_revealing_2020}. We showed that when applied to higher-resolution joint position data, the AR-HMM failed to identify interpretable behaviors, and we suggested an alternative model known as an SLDS instead, demonstrating that it performs better than the AR-HMM on a number of both quantitative and qualitative measures. We also proposed a novel framework for improving initializaton schemes for fitting SLDS models, though we leave the full implementation of this framework for future studies. In all, our findings raise important questions about the utility of different latent variable models for studying unconstrained behavior, as well as the technical features and characteristics of the data that are the most critical to include in such models going forward. For example, are there other methods for dealing with noise variability that would be superior to the SLDS, or which would permit better AR-HMM performance? Is it possible to obtain sufficient state representations of behavior using only postural data, or must we also include additional features such as the animal's velocity and/or head direction? How can we extend these models to incorporate more information about an animal's environment, such as sensory inputs (akin to the GLM-HMM framework in Chapter \ref{ch:glmhmm})? What alternative (non-linear) mappings might we consider that govern an animal's current and prior positions? These and other questions will all undoubtedly be active areas of exploration for future computational neuroethologists. 

Finally, in Chapter \ref{ch:bestlds} we combined elements of the previous two chapters to develop a novel approach for inferring the system parameters of Bernoulli LDS models, with a particular interest in applications to binary decision-making. We showed that our method, ``bestLDS", recovered accurate and consistent estimates of the system parameters and provided robust initializations for EM fitting that drastically sped up total computation time.  We also applied bestLDS to the same cognitive decision-making task as discussed in Chapter \ref{ch:glmhmm}. In addition to demonstrating our method's utility in real-world settings, our findings also revealed parallels between the descriptions of behavior uncovered by bestLDS (in which the latent states are continuous) and that of the GLM-HMM, which has discrete states. This begs the question: what is the true dynamical structure of decision-making behavior, and how might this be corroborated at the level of neural activity? In all, this work joins a longstanding history of work on parameter estimation methods for latent variable models \cite{martens_learning_2010, anandkumar_tensor_2014, belanger_linear_2015, hazan_learning_2017, hazan_spectral_2018} and fortifies the need for new lines of inquiry. How broadly can we extend these methods to different distribution classes? Can we prove theoretical guarantees on performance? Can we adapt the methods used for bestLDS to also work in the discrete-state GLM-HMM? We hope that this study will serve as a launch point for future work on estimation methods for latent variable models, data distributions, and applications that are of particular interest in neuroscience, especially as such models grow in popularity in the field.  

Overall, we think that the results presented in this thesis speak to the need to carefully consider the design of computational models that are used for quantifying complex mammalian behaviors. Without the right elements, we risk missing critical pieces of the puzzle as we aim to assemble a full picture of the brain-behavior relationship. A description of the cognitively demanding decision-making task in Chapter \ref{ch:glmhmm} that didn't include consideration of sensory inputs or possible neural pathway-specific effects on behavior would never have revealed the same insights about DMS-dependent decision-making strategies as we were able to uncover. Similar factors are likely relevant to uncover the dynamic structure underlying the exploratory behavior we discussed in Chapter \ref{ch:slds}, along with the critical technical details we highlighted in our model comparisons. Nor should we underestimate the need to develop computationally efficient mechanisms for fitting such models, as we did in Chapter \ref{ch:bestlds}, especially as rates of data acquisition grow.  

Of course, this thesis but scratches the surface and there is much that remains to be done. Perhaps the most compelling questions relevant to the work discussed here concern how to more directly incorporate neural activity into behavioral models, as in \cite{schneider_learnable_2023}; how to build hierarchical models that learn the dynamic structure of behavior (and its neural correlates) over multiple timescales, as in \cite{tao_statistical_2019}; and how to design composite models that don't require us to choose between continuous- and discrete-state representations of the brain-behavior relationship that are also meaningful and interpretable, with inspiration from a plethora of past and ongoing work \cite{escola_hidden_2011, wiltschko_mapping_2015, johnson_composing_2016, linderman_hierarchical_2019, calhoun_unsupervised_2019, roy_extracting_2021, costacurta_distinguishing_2022, ashwood_mice_2022, bolkan_opponent_2022, weinreb_keypoint-moseq_2023}. Building on these directions will no doubt open up new insights into previously inaccessible dimensions of the dynamic structure of complex behaviors. 
